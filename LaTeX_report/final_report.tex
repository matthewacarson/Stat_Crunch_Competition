\documentclass[12pt]{article}
\usepackage[english]{babel}
\title{StatCrunch Competition\\ Twitch Dataset}
\author{Matthew Carson\\ University of California, Los Angeles}
\date{\today}

\usepackage{amsmath}
\usepackage{url}
\usepackage{hyperref}
\usepackage{graphicx}
\usepackage{float} % for [H] placement specifier
\usepackage{wrapfig}
\usepackage{caption} % for customizing captions
\usepackage[margin=1in]{geometry} % for setting margins
\usepackage{setspace} % for adjusting line spacing
\usepackage{subcaption}
% Line spacing
\setstretch{1.25}
\usepackage[autostyle, english = american]{csquotes}
\MakeOuterQuote{"}

% Define indentation length
\newlength{\myindent}
\setlength{\myindent}{3em}

% Paragraph indentation
\setlength{\parindent}{\myindent}

%%%%%%%%%%%%%%%%%%%%%%%%%%%%%
% Begin Document
% Title Page
%%%%%%%%%%%%%%%%%%%%%%%%%%%%%
\begin{document}
\begin{titlepage}
\maketitle
\thispagestyle{empty} % Removes page number from the cover page
\end{titlepage}
% Set page numbering to roman for preliminary pages
\pagenumbering{roman}
%%%%%%%%%%%%%%%%%%%%%%%%%%%%%
% Table of Contents
%%%%%%%%%%%%%%%%%%%%%%%%%%%%%
\tableofcontents

%%%%%%%%%%%%%%%%%%%%%%%%%%%%%
% List of Tables
%%%%%%%%%%%%%%%%%%%%%%%%%%%%%
\listoftables

%%%%%%%%%%%%%%%%%%%%%%%%%%%%%
% List of Figures
%%%%%%%%%%%%%%%%%%%%%%%%%%%%%
\listoffigures
\newpage
%%%%%%%%%%%%%%%%%%%%%%%%%%%%%
% Begin body of report
%%%%%%%%%%%%%%%%%%%%%%%%%%%%%
% Set page numbering to arabic for main content
\pagenumbering{arabic}

\section{Summary Statistics}\

Since these data were not randomly sampled, it would be inappropriate to conduct inference (e.g., construct confidence intervals or conduct hypothesis tests). Because of this, it is not possible to estimate population parameters, that is, make claims or generalizations about the broader population of Twitch users. However, since these data represent the top 900 Twitch users, statistics can be calculated, and relationships can be discovered about that population.

Initial calculations were made before presenting the summary statistics. Since values for \texttt{`Watch time (mins)'} and \texttt{`Stream time (mins)'} were typically very large, requiring scientific notation to express, values were rescaled by dividing the values in both columns by the product of 60 (60 minutes in an hour) times 52 (number of weeks in a year) to make the numbers more manageable:

\begin{equation}
Mean\ weekly\ watch\ hours = \dfrac{Watch\ time\ (mins)}{60 \ast 52}
\end{equation}
and
\begin{equation}
Mean\ weekly\ stream\ hours = \dfrac{Stream\ time\ (mins)}{60 \ast 52}
\end{equation}
\newline
Additional statistics were calculated as well.\footnote{Entire dataset available: \href{https://www.statcrunch.com/app/index.html?dataid=4597814&token=OTI3Z8\%2F0N6hSC1KVw9hTXmyjHLnuZvqCMyuxkgn1QRYPhIQfDVLUFClF3Y41ShOi4C\%2BMKL5\%2FHgpBTXKukjWOPGD4pN\%2FCkiobeyKouIjPB7LoPvHOTDN7wUNtPZQd2\%2BNjOwAtSMQl9aKQrbthjCSuuSihsliiTo0MvakPDYN0lwiE8N11ITBSTS9QJH9QgHEmO4ahoV6IkASuVdKosV\%2FJSQ\%3D\%3D}{(click here)}}

\begin{itemize}
	\item \texttt{`Followers Prev Yr’} = \texttt{`Followers’} - \texttt{`Followers gained’}.
	\item \texttt{`Followers gained percent’} = \texttt{`Followers gained’} / \texttt{`Followers Prev Yr’}.
\end{itemize}

%%%%%%%%%%%%%%%%%%%%%%%%%%%%%
% Table: Summary Statistics
%%%%%%%%%%%%%%%%%%%%%%%%%%%%%
\begin{table}[H]
  \centering
  \includegraphics[width=0.8\linewidth]{../StatCrunch_Results/table}
  \captionsetup{justification=centering, singlelinecheck=false, margin=2cm}
  \caption[Summary Statistics]{Summary Statistics.}
  \label{table:summary_stats}
\end{table}

Because all distributions are heavily right-skewed (skewness $\geq$ 2.6), medians, represented with the Greek letter eta ($\eta$), are reported instead of means (all values are from Table \ref{table:summary_stats}). The majority of the top nine hundred accounts stream content at least 30 hours per week ($\eta \approx$ 34.23) and are watched more than 90 thousand hours per week $(\eta \approx$ 91,422). Most accounts gained a substantial number of followers from the previous year ($\eta \approx$ 66,003), which represents a median increase of approximately 16 percent. Because of the heavy skewness of the distributions, easy-to-interpret visualizations were difficult to make (Fig. \ref{fig:histogram_matrix}).

\section{Correlation}\

To assess the relationships between numeric variables, Spearman’s correlation coefficients were calculated (Fig. \ref{fig:spearman_correlogram}). Because of the non-linearity of the relationships (Fig. \ref{fig:stream_scatter_matrix}, Fig. \ref{fig:watch_scatter_matrix}, Fig. \ref{fig:follow_gain_scatter_matrix}), typical Pearson’s R correlation coefficients would be inappropriate. Spearman’s correlation coefficients are preferred for assessing the strength of non-linear relationships.
%%%%%%%%%%%%%%
% Figure: Spearman Correlogram 
%%%%%%%%%%%%%%
\begin{figure}[H]
  \centering % width=\linewidth, height=0.4\textheight
  \includegraphics[width=\linewidth]{../StatCrunch_Results/spearman_correlogram}
  \captionsetup{justification=centering, singlelinecheck=false, margin=2cm}
  \caption[Spearman Correlogram]{Spearman Correlogram}
  \label{fig:spearman_correlogram}
\end{figure}

The relationships between numeric variables are surprising, especially the absence of some correlations where one would think they would exist (Fig. \ref{fig:spearman_correlogram}). \texttt{`Stream time’} has a moderately strong negative Spearman correlation ($\rho$) with \texttt{`Average viewers’} ($\rho = -0.63$), which is counterintuitive given that one might expect more frequent streaming to result in more viewers, but that is not the case. In terms of change over time, accounts that streamed more hours did \emph{not} gain more followers; indeed, the accounts that were in the top decile of weekly stream time gained less than one-fifth of the followers that the accounts in the lowest stream time decile gained (Table \ref{table:followers_gained_stream_decile_table}; Fig. \ref{fig:followers_gained_stream_deciles}).  With respect to surprising absences of relationships, \texttt{`Stream time’} has practically no relationship with \texttt{`Watch time’} ($\rho = 0.07$; Fig. \ref{fig:spearman_correlogram}). Together, these findings suggest that a strategy of merely increasing one’s streaming time does not “pay off” in terms of the number of followers or viewers.

%%%%%%%%%%%%%%%%%%%%%%%%%%%%%
% Table: followers_gained_stream_decile_table
%%%%%%%%%%%%%%%%%%%%%%%%%%%%%
\begin{table}[H]
  \centering % width=0.8\linewidth, height=0.35\textheight
  \includegraphics[width=0.8\linewidth]{../StatCrunch_Results/followers_gained_stream_deciles/table}
  \captionsetup{justification=centering, singlelinecheck=false, margin=2cm}
  \caption[Followers Gained by Stream Time Deciles]{Followers Gained by Stream Time Deciles. (The lowest decile streamed the least.)}
  \label{table:followers_gained_stream_decile_table}
\end{table}


%%%%%%%%%%%%%%%%%%%%%%%%
% Figure: followers_gained_stream_deciles
%%%%%%%%%%%%%%%%%%%%%%%%
\begin{figure}[H]
  \centering % width=0.8\linewidth, height=0.4\textheight
  \includegraphics[width=0.6\linewidth]{../StatCrunch_Results/followers_gained_stream_deciles/barplot}
  \captionsetup{justification=centering, singlelinecheck=false, margin=2cm}
  \caption[Followers Gained by Stream Time Deciles]{Followers Gained by Stream Time Deciles. (The lowest decile streamed the least.)}
  \label{fig:followers_gained_stream_deciles}
\end{figure}

\section{Simple Linear Regressions}\

Because the relationships between the variables were not linear, it was difficult to fit models using simple linear regression. However, using transformations, I was able to fit some models.

\subsection{Stream Hours to Predict Average Viewers}\

Do accounts that stream more hours have more viewers? Intuition would suggest yes, but the question deserves empirical examination. A simple linear regression was run to assess the strength between \texttt{`Mean weekly stream hours’} (independent variable) and \texttt{`average viewers’} (dependent variable). Because of the non-linear relationship between the variables, the residuals were highly non-normal. An inverse (reciprocal) transformation of \texttt{`Average viewers’} was performed to correct for the non-normality of the residuals. The transformation greatly improved the distribution of the residuals, making them nearly normal. The transformation could not correct for heteroskedasticity, but this is not an issue since inference is not being conducted.

\subsubsection{Model Specification}

The regression model is as follows:

\begin{equation}
\dfrac{1}{Average\ viewers_{i}} = \beta_{0} + \beta_{1} \ast Mean\ weekly\ stream\ hours_{i} 
\end{equation}
where \textit{i} is a Twitch account. \textit{`Average viewers'} is the average number of viewers that watched the respective Twitch account; \textit{`Mean weekly stream hours'} is the number of hours that the respective Twitch account streamed over the year divided by 52.

\subsubsection{Simple Linear Regression Results}\

%The regression results are reported below (Table \ref{table:reciprocal_regression_table}).

%%%%%%%%%%%%%%%%%%%%%%%%%%%%%
% Table: Regression table
%%%%%%%%%%%%%%%%%%%%%%%%%%%%%
\begin{table}[!ht]
  \centering % width=0.8\linewidth, height=0.375\textheight
  \includegraphics[scale=0.75]{../StatCrunch_Results/reciprocal/regression_table}
  \captionsetup{justification=raggedright, singlelinecheck=false, margin=2cm}
  \caption[Regression: Average Viewers Predicted by Stream Hours]{Simple linear regression model showing a moderate relationship between average viewers and mean weekly stream hours.}
  \label{table:reciprocal_regression_table}
\end{table}

R-squared is moderate (0.56), suggesting that the mean weekly stream hours can explain 56 percent of the variation in the average number of viewers. Because of the inverse transformation of the dependent variable, the signs of the intercept and mean weekly stream hours coefficient are reversed. This makes sense, though, since an increase in the denominator of the equation (when the right-hand side of the equation is back transformed; Equation \ref{eq:back_transformed_reciprocal_eq}) will diminish the predicted number of average viewers.

%\begin{wraptable}{l}{0.5\textwidth} % Adjust the width as needed
%%    \centering
%    \includegraphics[width=\linewidth]{../StatCrunch_Results/reciprocal/regression_table}
%    \captionsetup{justification=centering, singlelinecheck=false, margin=0cm}
%    \caption[Regression: Average Viewers Predicted by Stream Hours]{Simple linear regression model showing a moderate relationship between average viewers and mean weekly stream hours.}
%    \label{table:reciprocal_regression_table}
%\end{wraptable}

Figure \ref{fig:stream_time_reciprocal_avg_viewers} shows the relationship between the mean weekly stream hours and the inverse of average viewers. The other scatter plots show the residuals. The histogram and Q-Q plots show that the residuals are now nearly normal, although excess kurtosis is still high, making the distribution of the residuals leptokurtic (skewness = -0.57036091; excess kurtosis = 6.0727871). Removing extreme dependent variable observations improves this, but it is unlikely to be helpful since inference is not being conducted, and fitting data to your model is not the best practice.\footnote{It is better to adjust your model to fit the data than the other way around.} Heteroskedasticity also was not corrected, but it is not an issue as no hypothesis testing is being conducted, nor are confidence intervals being calculated. The model with coefficients is as follows:
\begin{equation}
 \dfrac{1}{Average\ viewers_{i}} = 4.04e^{-5} + 9.15e^{-6} \ast Mean\ weekly\ stream\ hours_{i} \label{eq:reciprocal_model_w_coef}
\end{equation}

\subsubsection{Scatter Plots}\

%%%%%%%%%%%%%%%%%%%%%%%%
% stream_time_reciprocal_avg_viewers
%%%%%%%%%%%%%%%%%%%%%%%%
\begin{figure}[H]
  \centering % width=\linewidth, height=\textheight
  \includegraphics[scale=0.5]{../StatCrunch_Results/reciprocal/stream_time_reciprocal_avg_viewers_plots}
  \captionsetup{justification=centering, singlelinecheck=false, margin=2cm}
  \caption[Average Viewers Predicted by Stream Hours]{Model plots. The residuals are symmetric, making the model a decent fit, notwithstanding the non-constant variance.}
  \label{fig:stream_time_reciprocal_avg_viewers}
\end{figure}

\noindent Since one usually does not have a particular interest in knowing what the reciprocal of the average number of viewers is, \texttt{`Average viewers'} was reverse transformed, and the transformed equation was plotted over the data on their original scale (Fig. \ref{fig:transformed_reciprocal_scatter}).


\begin{figure}[H]
  \centering % width=\linewidth, height=\textheight
  \includegraphics[width=0.8\linewidth]{../StatCrunch_Results/reciprocal/scatter_plot_reverse_transformed}
  \captionsetup{justification=centering, singlelinecheck=false, margin=2cm}
  \caption[Back Transformed Average Viewers Predicted by Stream Hours]{Back Transformed Average Viewers Predicted by Stream Hours. The fitted line is Equation \ref{eq:back_transformed_reciprocal_eq}}.
  \label{fig:transformed_reciprocal_scatter}
\end{figure}

\begin{equation}
Average\ viewers_{i} = \dfrac{1}{4.04e^{-5} + 9.15e^{-6} \ast Mean\ weekly\ stream\ hours_{i}} \label{eq:back_transformed_reciprocal_eq}
\end{equation}

\subsubsection{Discussion}\

Contrary to what one might think, the amount of time that a Twitch account streams has a negative association with the average number of viewers. So, if one wants to increase their viewership, streaming more does not seem to be a good strategy. A quick illustration is helpful. The median weekly stream hours for all accounts is 34.23 hours per week. Using Equation \ref{eq:back_transformed_reciprocal_eq}, we can estimate that the mean average number of viewers\footnote{The formulation appears awkward, but the variable is \texttt{`Average viewers'}, and the predicted value represents the mean \texttt{`Average viewers'} for that particular \texttt{x} value.} for accounts that stream the median number of hours per week is $\approx$ 2826. Accounts that stream ten hours less than the median per week have a mean average viewership of 3813, which is 987 more viewers than accounts that stream the median number of hours per week (See Table \ref{tab:prediction_matrix} for predicted values). 

\subsection{Stream Hours to Predict Followers}\

Do accounts that stream more have more followers? Intuitively, it would make sense. A simple linear regression was run to assess the relationship between streaming time, \texttt{`Mean weekly stream hours'}, and the number of \texttt{`Followers'} Twitch accounts have. The initial model revealed a highly non-linear relationship with highly non-normal residuals. A log transformation of \texttt{`Followers'} was conducted to correct for this and the models subsequently fit the data much better.

\subsubsection{Model Specification}
The regression model is as follows:

\begin{equation}
	ln(Followers_{i}) = \beta_{0} + \beta_{1} \ast Mean\ weekly\ stream\ hours_{i}
	\label{eq:ln_followers_stream}
\end{equation}
where \emph{i} is a Twitch account; \texttt{`Followers'} is the number of followers that a Twitch account has; \texttt{`Mean weekly stream hours'} is the amount of time a Twitch account streamed over the year divided by 60 and then divided by 52.

\subsubsection{Simple Linear Regression Results}\

The regression reveals a rather weak correlation between the mean weekly stream hours and the number of followers that an account has. R-squared is only 0.056. The coefficient for \texttt{`Mean weekly stream hours'} (Table \ref{tab:ln_followers_stream}) was exponentiated ($\emph{e}^{-0.0088658957} = 0.99117$), indicating that for every additional weekly streaming hour, an account should expect 0.88 percent fewer followers. (See Table \ref{tab:ln_followers_stream} for all model parameters.)

\subsubsection{Scatter Plots}

\begin{figure}[H]
\centering
	\includegraphics[height=0.7\textheight]{../StatCrunch_Results/ln_followers_stream/plot}
	\captionsetup{justification=centering, singlelinecheck=false, margin=2cm}
	\caption[Residual Plots: Followers by Stream Time]{\texttt{`Followers'} regressed against \texttt{`Mean weekly stream hours'}. }
	\label{fig:ln_followers_stream_plot}
\end{figure}

\begin{figure}[H]
\centering
	\includegraphics[scale=0.65]{../StatCrunch_Results/ln_followers_stream/scatter_small}
	\captionsetup{justification=centering, singlelinecheck=false, margin=2cm}
	\caption[Scatter Plot: Followers by Stream Time]{\texttt{`Followers'} regressed against \texttt{`Mean weekly stream hours'}. }
	\label{fig:ln_followers_stream_scatter}
\end{figure}

\subsubsection{Discussion}\

The results, again, are counterintuitive. Rather than observing more followers with accounts that stream more frequently, we see the opposite trend, albeit weakly. Again, this suggests that simply streaming more frequently or for longer durations is not sufficient to build a more successful Twitch account.


\begin{table}[H]
\centering
  \includegraphics[height=0.5\textheight]{../StatCrunch_Results/ln_followers_stream/table}
  \captionsetup{justification=centering, singlelinecheck=false, margin=2cm}
  \caption[Simple Linear Regression for Followers by Stream Time]{\texttt{`Followers'} regressed against \texttt{`Mean weekly stream hours'}}
  \label{tab:ln_followers_stream}
\end{table}

\section{Multiple Linear Regression}

\subsection{Mature Accounts and Followers}\

How much does the difference in watch time explain the difference in the number of followers for accounts categorized as mature or not mature? Twitch accounts classified as \texttt{`Mature'} have fewer followers than those not classified as \texttt{`Mature'}. At the same time, mature accounts are watched fewer hours per week (Figure \ref{fig:box_plots_mature}).\footnote{Mature accounts also tend to have fewer peak and average viewers, and they also have not gained as many followers over the period. See Figure \ref{fig:box_plots_mature}.} While it is clear that mature accounts are watched less, it is interesting to ask how much that disparity in watch time can "explain"\footnote{This is observational data, so causal claims cannot be made here.} the disparity in the number of followers.

\subsubsection{Model Specificiation}\

Since \texttt{`Followers'} is highly skewed, the log of \texttt{`Followers'} was taken. A difference in means of the log followers was calculated. The difference in log means (-0.13351611) is exponentiated to get the ratio (0.875) between the two means (Equation \ref{eq:log_means_followers_mature}). Thus, mature Twitch accounts have 12.5 percent fewer followers than accounts that are not classified as mature.

\begin{equation}
\begin{aligned}
\mu_{1} & = \text{Mean of } \ln(\text{Followers}) \text{ where Mature} \\
\mu_{2} & = \text{Mean of } \ln(\text{Followers}) \text{ where not Mature} \\
\mu_{1} - \mu_{2} & = -0.13351611 \\
\emph{e}^{-0.13351611} & = 0.875
\end{aligned}
\label{eq:log_means_followers_mature}
\end{equation}

Next a multiple linear regression model was run to assess how much differences in the number of watch hours can explain the differences in the number of followers between mature and non-mature accounts. Log transformations on both the independent (\texttt{`Mean weekly watch hours'}) and dependent variable (\texttt{`Followers'}) were conducted to correct for the non-linear nature of the relationship. The model is as follows:

\begin{equation}
ln(Followers_{i}) = \beta_{0} + \beta_{1} \ast Mature_{i} + \beta_{2} \ast ln(Mean\ weekly\ watch\ hours_{i})
\label{eq:multi_linear_model}
\end{equation}

where \emph{i} is each Twitch account in the dataset; \texttt{`Followers'} is the number of followers; \texttt{`Mature'} is a binary variable where \texttt{1} is when the account has been classified as mature, and \texttt{0} is when it has not; \texttt{`Mean weekly watch hours'} is the total watch time in minutes for a Twitch account divided by 60 and divided by 52.

\subsubsection{Multiple Linear Regression Results}\

\begin{table}[H]
  \centering % width=\linewidth, height=\textheight
  \includegraphics[scale = 0.7]{../StatCrunch_Results/followers_mature_watch_hrs/multi_regression_table}
  \captionsetup{justification=centering, singlelinecheck=false, margin=2cm}
  \caption[Multiple Linear Regression for Followers by Mature]{Followers regressed against Mature and Mean weekly watch hours}
  \label{tab:multi_regression_tab}
\end{table}

R-squared is rather weak (0.25), but the difference in watch time (mean weekly watch hours) does explain some of the differences observed between mature and non-mature accounts. The coefficient for \texttt{`Mature'} is -0.080855823. Exponentiating the coefficient ($\emph{e}^{-0.080855823}$) results in 0.9223. This is the ratio of followers for mature accounts to non-mature accounts when the natural log of mean weekly watch hours is held constant. In other words, mature accounts have 7.7 (1- 0.9223) percent fewer followers than non-mature accounts when controlling for differences in watch time.

\begin{figure}
    \centering
    \begin{subfigure}{0.45\textwidth}
        \centering
        \includegraphics[width=\textwidth]{../StatCrunch_Results/followers_mature_watch_hrs/residual_histogram}
        \caption{Histogram: residuals are nearly normal}
        \label{fig:plot1}
    \end{subfigure}
    \hfill
    \begin{subfigure}{0.45\textwidth}
        \centering
        \includegraphics[width=\textwidth]{../StatCrunch_Results/followers_mature_watch_hrs/residual_plot}
        \caption{No non-linear trends detected in the data}
        \label{fig:plot2}
    \end{subfigure}
    \caption[Residual Plot for ML Regression]{Plots of residuals for the multiple linear regression model in Equation \ref{eq:multi_linear_model} and Table \ref{tab:multi_regression_tab}.}
    \label{fig:both_plots}
\end{figure}

\subsubsection{Discussion}\

The results suggest that the difference between the number of followers that mature and non-mature accounts have is only slightly "explained" by differences in the number of mean weekly stream hours between the two types of accounts. In a way, this is not totally surprising. Figure \ref{fig:spearman_correlogram} shows that there is only a moderate correlation between watch time and the number of followers an account has. It could be that there is some other unobserved variable out there that explains the difference in followers, or, perhaps just as likely, mature accounts are less popular in general and struggle to gain as many followers. Of course, identifying a mechanism like this is beyond the scope of this paper, as that would require more data.

\section{Logistic Regression}

\subsection{Predicting Partnered Status}\

Of the variables in the dataset, is the number of followers the best predictor of whether a Twitch account is partnered? To answer this, I ran a logistic regression on each of the predictor values in the dataset. Using the log-likelihood from the output, I calculated the Akaike information criterion (AIC), which estimates the quality of a statistical model relative to other models. The formula for AIC is as follows:

\begin{equation}
AIC = 2k -2 \ast ln(Likelihood)
\end{equation}
where \emph{k} is the number of parameters. When comparing across models with the same number of predictor variables, k remains constant; thus, the model with the greatest ln(likelihood) is the model with the lowest AIC. The model with the smallest AIC is usually preferred over more complicated models unless there is a good theoretical reason to include those additional variables.\footnote{To be clear, one should be guided by theory first; AIC is only a tool.}

\subsubsection{AIC Values}

After running each model, \texttt{`Followers'} was the best model to predict \texttt{`Partnered'} status (AIC = 4 - 2 $\times$ -116.78416 = 237.5683).
\newline
\newline
\noindent AIC values for other models with a single predictor variable:

\noindent Variable(s) \hfill AIC 
\hspace{0.4em}
\hrule
\begin{itemize}
	\item \texttt{`Peak viewers'} \dotfill 245.79
	\item \texttt{`Average viewers'} \dotfill 246.42
	\item \texttt{`Mature'} \dotfill 246.22
	\item \texttt{`Mean weekly watch hours'} \dotfill 245.77
	\item \texttt{`Mean weekly stream hours'} \dotfill 246.36
\end{itemize}

\noindent Adding additional variables beyond \texttt{`Followers'} did not improve the AIC:

\noindent Variable(s) \hfill AIC 
\hspace{0.4em}
\hrule
\begin{itemize}
    \item \texttt{`Followers + Peak viewers'} \dotfill 238.80
    \item \texttt{`Followers + Average viewers'} \dotfill 237.67
    \item \texttt{`Followers + Mature'} \dotfill 239.07
    \item \texttt{`Followers + Mean weekly watch hours'} \dotfill 239.07
    \item \texttt{`Followers + Mean weekly stream hours'} \dotfill 238.32
\end{itemize}
Thus, I opted to only use \texttt{`Followers'} to predict partnered status in my logit model.

\subsubsection{Model Specification}\

I decided to rescale \texttt{`Followers'} by dividing it by 100,000 so that the new variable is 100,000 followers. This will make the coefficient easier to interpret.
\begin{equation}
\begin{aligned}
	ln(\frac{p_{i}}{1-p_{i}}) & = logit(Partnered_{i})  = \beta_{0} + \beta_{1} \ast Followers/100,000_{i} \\
	\frac{p_{i}}{1-p_{i}} & = odds
\end{aligned}
\end{equation}
where \emph{i} is a Twitch account; \texttt{`Partnered'} is a binary variable indicating whether the respective account has attained partnered status or not; \texttt{`Followers / 100,000'} is the number of followers that an account has; \emph{p} is the probability of an account being achieving partnered status.

\subsubsection{Logistic Regression Results}\

The logistic regression shows a positive relationship between the number of followers a Twitch account has and the log odds of being a partnered account (Table \ref{tab:logistic_table}). Exponentiating the coefficient for \texttt{`Followers / 100,000'} (\emph{e}$^{0.10616385}$)  results in 1.112, which means that for every 100,000 followers that an account has, the odds of being a partnered account increase 11.2 percent.

\begin{table}[H]
  \centering % width=\linewidth, height=\textheight
  \includegraphics[scale=1]{../StatCrunch_Results/logit_partnered_followers/table}
  \captionsetup{justification=centering, singlelinecheck=false, margin=2cm}
  \caption[Logistic Regression: Predict Partnered Status by Number of Followers]{Logistic Regression Table. The results indicate a positive relationship between the number of followers an account has and whether it has achieved partnership status.}
  \label{tab:logistic_table}
\end{table}

\subsubsection{Figures}\

A scatter plot was generated to visualize the relationship. Since all values for \texttt{`Partnered'} take either a zero or one, random noise using the normal distribution simulation function centered on zero with a standard deviation of 0.01 --- $\mathcal{N}(0, 0.01)$ --- was generated in StatCrunch to make visualization clearer. Then, the absolute value of that simulated normal distribution was either added or subtracted from the \texttt{`Partnered'} value depending on if the value was a one or zero, respectively. 

\noindent The equation representing the generation of the \texttt{`PartneredNoise'} variable is:

\begin{equation}
\text{PartneredNoise} = 
\begin{cases}
    \text{Partnered} + \lvert \text{Normal} \rvert & \text{if } \text{Partnered} = 1 \\
    \text{Partnered} - \lvert \text{Normal} \rvert & \text{if } \text{Partnered} = 0
\end{cases}
\end{equation}

\noindent where \texttt{`Partnered'} is the binary variable for whether an account is partnered; \texttt{`Normal'} is the simulated normal distribution based on $\mathcal{N}(0, 0.01)$.

The model was exponentiated and plotted over the scatter plot. It represents the probability that an account will be partnered, as predicted by the number of followers that that account has.

\begin{equation}
\hat{p_{i}} = \frac{e^{2.8464312 + 0.10616385 \ast (Followers/100,000)}}{1 + e^{2.8464312 + 0.10616385 \ast (Followers/100,000)}}
\end{equation}

This resulted in the following plot:

\begin{figure}[H]
\centering
	\includegraphics[height=0.52\textheight]{../StatCrunch_Results/logit_partnered_followers/scatter_plot}
	\captionsetup{justification=centering, singlelinecheck=false, margin=2cm}
	\caption[Scatter Plot: Partnered Status Predicted by Number of Followers]{Scatter Plot: Partnered Status Predicted by Number of Followers.}
	\label{fig:logit_partnered_followers}
\end{figure}

\subsubsection{Goodness of Fit}\

The Hosmer-Lemeshow Goodness-of-Fit observed vs. expected successes for partnered status suggest that, in general, the model is reasonably calibrated (Table \ref{tab:logistic_table}). The Hosmer-Lemeshow Goodness-of-Fit Test assesses how accurately the model predicts the success rate in each decile of the independent variable (in this case, \texttt{`Followers'}). The model predicts successful partnered status reasonably well, but it is more limited in its ability to predict non-partnered status (failures). This could be because of the extreme class imbalance in the data (success-to-failure ratio of 873:27).

\subsubsection{Discussion}\

The number of followers is the best predictor of the probability that an account will be partnered. However, this analysis is hampered because of the severe class imbalance in the data. That notwithstanding, the disparity in followers between partnered and non-partnered accounts is striking. 123 of the 873 partnered accounts have more followers than the maximum number (1714324) of followers that non-partnered accounts have.\footnote{Author's calculations in StatCrunch.} Future analyses could randomly sample Twitch data using a blocking technique to ensure that success and failure conditions are more equally represented in the data.


\section{Conclusion}

The Twitch dataset analysis stands out primarily due to the unexpected findings it yielded. There is practically no relationship between the amount of time that an account streams and the amount of time that the account is watched. And contrary to what one might expect, streaming more frequently has a strong \emph{negative} relationship with the average number of viewers an account has. In short, if you stream more, fewer people are watching you (See Figure \ref{fig:spearman_correlogram} and 3.1: Stream Hours to Predict Average Viewers). Similarly, the number of stream hours has a negative relationship with the number of followers that an account has (see 3.2: Stream Hours to Predict Followers). 












%%%%%%%%%%%%%%%%%%%%%%%%%%
%%%%%%%%%%%%%%%%%%%%%%%%%%
%%%%%%%%    Appendix    %%%%%%%%%%%
%%%%%%%%%%%%%%%%%%%%%%%%%%
%%%%%%%%%%%%%%%%%%%%%%%%%%
\newpage 

\section{Appendix}

\subsection{Additional Figures}

\begin{figure}[H]
  \centering % width=\linewidth, height=\textheight
  \includegraphics[width=0.8\linewidth]{../StatCrunch_Results/mature/box_plots}
  \captionsetup{justification=centering, singlelinecheck=false, margin=2cm}
  \caption[Box Plots of Twitch Accounts by Mature Calssification]{Box Plots of Twitch Accounts by Mature Calssification}
  \label{fig:box_plots_mature}
\end{figure}



%%%%%%%%%%%%%%%%%%%
% Figure: Stream Hours Scatter Plot Matrix
%%%%%%%%%%%%%%%%%%%
\begin{figure}[H]
  \centering
  \includegraphics[width=0.795\linewidth]{../StatCrunch_Results/stream_scatter_plot_matrix.png}
  \captionsetup{justification=centering, singlelinecheck=false, margin=2cm}
  \caption[Stream Hours Scatter Plot Matrix]{Relationships between \texttt{`Mean weekly stream hours'} and other variables.}
  \label{fig:stream_scatter_matrix}
\end{figure}

%%%%%%%%%%%%%%%%%%%%%%%%%%%%%%%%%%%%%%%%%%%%
% Figure: Watch Hours Scatter Plot Matrix
%%%%%%%%%%%%%%%%%%%%%%%%%%%%%%%%%%%%%%%%%%%%
\begin{figure}[H]
  \centering
  \includegraphics[width=0.8\linewidth]{../StatCrunch_Results/watch_scatter_plot_matrix.png}
  \captionsetup{justification=centering, singlelinecheck=false, margin=2cm}
  \caption[Watch Hours Scatter Plot Matrix]{Relationships between \texttt{`Mean weekly watch hours'} and other variables.}
  \label{fig:watch_scatter_matrix}
\end{figure}

%%%%%%%%%%%%%%%%%%%%%%%%
% Figure: Followers Gained Scatter Plot Matrix
%%%%%%%%%%%%%%%%%%%%%%%%
\begin{figure}[H]
  \centering
  \includegraphics[width=0.8\linewidth]{../StatCrunch_Results/follow_gain_scatter_matrix}
  \captionsetup{justification=centering, singlelinecheck=false, margin=2cm}
  \caption[Followers Gained Scatter Plot Matrix]{Relationships between \texttt{`Followers gained'} and other variables.}
  \label{fig:follow_gain_scatter_matrix}
\end{figure}

%%%%%%%%%%%%%
% Figure: Histogram Matrix
%%%%%%%%%%%%%
\begin{figure}[H]
  \centering
  \includegraphics[width=0.8\linewidth]{../StatCrunch_Results/Histogram_Matrix.png}
  \captionsetup{justification=centering, singlelinecheck=false, margin=2cm}
  \caption[Histogram Matrix]{All distributions are heavily skewed and non-normal.}
  \label{fig:histogram_matrix}
\end{figure}


%%%%%%%%%%%%%
% Figure: Prediction Matrix
%%%%%%%%%%%%%
\begin{table}[H]
  \centering
  \includegraphics[scale=1]{../StatCrunch_Results/reciprocal/prediction_matrix}
  \captionsetup{justification=centering, singlelinecheck=false, margin=2cm}
  \caption[Prediction Matrix]{Predicted Values.}
  \label{tab:prediction_matrix}
\end{table}

\subsection{StatCrunch Dataset}\

\begin{tiny}
\url{https://www.statcrunch.com/app/index.html?dataid=4597814&token=OTI3Z8\%2F0N6hSC1KVw9hTXmyjHLnuZvqCMyuxkgn1QRYPhIQfDVLUFClF3Y41ShOi4C\%2BMKL5\%2FHgpBTXKukjWOPGD4pN\%2FCkiobeyKouIjPB7LoPvHOTDN7wUNtPZQd2\%2BNjOwAtSMQl9aKQrbthjCSuuSihsliiTo0MvakPDYN0lwiE8N11ITBSTS9QJH9QgHEmO4ahoV6IkASuVdKosV\%2FJSQ\%3D\%3D}
\end{tiny}

% ceil(Rank(Mean weekly stream hours) / 90)

\end{document}