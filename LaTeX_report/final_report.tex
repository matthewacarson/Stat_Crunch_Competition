\documentclass[12pt]{article}
\usepackage[english]{babel}
\title{StatCrunch Competition\\ Twitch Dataset}
\author{Matthew Carson\\ University of California, Los Angeles}
\date{\today}

\usepackage{amsmath}
\usepackage{url}
\usepackage{hyperref}
\usepackage{graphicx}
\usepackage{float} % for [H] placement specifier
\usepackage{wrapfig}
\usepackage{caption} % for customizing captions
\usepackage[margin=1in]{geometry} % for setting margins
\usepackage{setspace} % for adjusting line spacing
\usepackage{subcaption}
% Line spacing
\setstretch{1.25}
\usepackage[autostyle, english = american]{csquotes}
\MakeOuterQuote{"}

% Define indentation length
\newlength{\myindent}
\setlength{\myindent}{3em}

% Paragraph indentation
\setlength{\parindent}{\myindent}

%%%%%%%%%%%%%%%%%%%%%%%%%%%%%
% Begin Document
% Title Page
%%%%%%%%%%%%%%%%%%%%%%%%%%%%%
\begin{document}
\begin{titlepage}
\maketitle
\thispagestyle{empty} % Removes page number from the cover page
\end{titlepage}
% Set page numbering to roman for preliminary pages
\pagenumbering{roman}
%%%%%%%%%%%%%%%%%%%%%%%%%%%%%
% Table of Contents
%%%%%%%%%%%%%%%%%%%%%%%%%%%%%
\tableofcontents

%%%%%%%%%%%%%%%%%%%%%%%%%%%%%
% List of Tables
%%%%%%%%%%%%%%%%%%%%%%%%%%%%%
\listoftables

%%%%%%%%%%%%%%%%%%%%%%%%%%%%%
% List of Figures
%%%%%%%%%%%%%%%%%%%%%%%%%%%%%
\listoffigures
\newpage
%%%%%%%%%%%%%%%%%%%%%%%%%%%%%
% Begin body of report
%%%%%%%%%%%%%%%%%%%%%%%%%%%%%
% Set page numbering to arabic for main content
\pagenumbering{arabic}

\section{Summary Statistics}\

Since these data were not randomly sampled, it would be inappropriate to conduct inference (i.e., construct confidence intervals or conduct hypothesis tests). Because of this, it is not possible to estimate population parameters; that is, make claims or generalizations about the broader population of Twitch users. However, since these data represent the top 900 Twitch users, statistics can be calculated and relationships can be discovered about that population.

Initial calculations were made before presenting the summary statistics. Since values for \texttt{`Watch time (mins)'} were typically very large, requiring scientific notation to express, values were rescaled to \texttt{`Mean weekly watch hours’} by dividing \texttt{`Watch time (mins)'} by the product of 60 times 52 (number of weeks in a year) to make the numbers more manageable:

\begin{equation}
Mean\ weekly\ watch\ hours = \dfrac{Watch\ time\ (mins)}{60 \ast 52}
\end{equation}
\newline
Additional statistics were calculated as well:

\begin{itemize}
	\item \texttt{`Followers Prev Yr’} = \texttt{`Followers’} - \texttt{`Followers gained’}.
	\item \texttt{`Followers gained percent’} = \texttt{`Followers gained’} / \texttt{`Followers Prev Yr’}.
\end{itemize}

%%%%%%%%%%%%%%%%%%%%%%%%%%%%%
% Table: Summary Statistics
%%%%%%%%%%%%%%%%%%%%%%%%%%%%%
\begin{table}[H]
  \centering
  \includegraphics[width=0.8\linewidth]{../StatCrunch_Results/table}
  \captionsetup{justification=centering, singlelinecheck=false, margin=2cm}
  \caption[Summary Statistics]{Summary Statistics.}
  \label{table:summary_stats}
\end{table}

Because all distributions are heavily right skewed (skewness $\geq$ 2.6), medians, represented with Greek letter eta ($\eta$), are reported instead of means (all values are from Table \ref{table:summary_stats}). The majority of the top nine hundred accounts stream content at least 30 hours per week ($\eta \approx$ 34.23) and are watched more than 90 thousand hours per week $(\eta \approx$ 91,422). Most accounts gained a substantial number of followers from the previous year ($\eta \approx$ 66,003), which represents a median increase of approximately 16 percent. Because of the heavy skewness of the distributions, easy-to-interpret visualizations were difficult to make (Fig. \ref{fig:histogram_matrix}).

\section{Correlation}

To assess the relationships between numeric variables, Spearman’s correlation coefficients were calculated (Fig. \ref{fig:spearman_correlogram}). Because of the non-linearity of the relationships (Fig. \ref{fig:stream_scatter_matrix}, Fig. \ref{fig:watch_scatter_matrix}, Fig. \ref{fig:follow_gain_scatter_matrix}), typical Pearson’s R correlation coefficients would be inappropriate. Spearman’s correlation coefficients are preferred for assessing the strength of non-linear relationships.
%%%%%%%%%%%%%%
% Figure: Spearman Correlogram 
%%%%%%%%%%%%%%
\begin{figure}[H]
  \centering % width=\linewidth, height=0.4\textheight
  \includegraphics[width=\linewidth]{../StatCrunch_Results/spearman_correlogram}
  \captionsetup{justification=centering, singlelinecheck=false, margin=2cm}
  \caption[Spearman Correlogram]{Spearman Correlogram}
  \label{fig:spearman_correlogram}
\end{figure}

The relationships between numeric variables are surprising, especially the absence of some correlations where one would think they they would exist (Fig. \ref{fig:spearman_correlogram}). \texttt{`Stream time’} has a moderately strong negative Spearman correlation ($\rho$) with \texttt{`Average viewers’} ($\rho = -0.63$), which is counterintuitive given that one might expect more frequent streaming to result in more viewers, but that is not the case. In terms of change over time, accounts that streamed more hours did \emph{not} gain more followers; indeed, the accounts that were in the top decile of weekly stream time gained less than one-fifth of the followers that the accounts in the lowest stream time decile gained (Table \ref{table:followers_gained_stream_decile_table}; Fig. \ref{fig:followers_gained_stream_deciles}).  With respect to surprising absences of relationships, \texttt{`Stream time’} has practically no relationship with \texttt{`Watch time’} ($\rho = 0.07$; Fig. \ref{fig:spearman_correlogram}). Together, these findings suggest that a strategy of merely increasing one’s streaming time does not “pay off” in terms of the number of followers or viewers.

%%%%%%%%%%%%%%%%%%%%%%%%%%%%%
% Table: followers_gained_stream_decile_table
%%%%%%%%%%%%%%%%%%%%%%%%%%%%%
\begin{table}[H]
  \centering % width=0.8\linewidth, height=0.35\textheight
  \includegraphics[width=0.8\linewidth]{../StatCrunch_Results/followers_gained_stream_deciles/table}
  \captionsetup{justification=centering, singlelinecheck=false, margin=2cm}
  \caption[Followers Gained by Stream Time Deciles]{Followers Gained by Stream Time Deciles. (The lowest decile streamed the least.)}
  \label{table:followers_gained_stream_decile_table}
\end{table}


%%%%%%%%%%%%%%%%%%%%%%%%
% Figure: followers_gained_stream_deciles
%%%%%%%%%%%%%%%%%%%%%%%%
\begin{figure}[H]
  \centering % width=0.8\linewidth, height=0.4\textheight
  \includegraphics[width=0.6\linewidth]{../StatCrunch_Results/followers_gained_stream_deciles/barplot}
  \captionsetup{justification=centering, singlelinecheck=false, margin=2cm}
  \caption[Followers Gained by Stream Time Deciles]{Followers Gained by Stream Time Deciles. (The lowest decile streamed the least.)}
  \label{fig:followers_gained_stream_deciles}
\end{figure}

\section{Simple Linear Regressions}\

Because the relationships between the variables were not linear, it was difficult to fit models using simple linear regression. However, using transformations, I was able to fit some models.

\subsection{Stream Hours to Predict Average Viewers}\

Do accounts that stream more hours have more viewers? Intution would suggest yes, but the question deserves empirical examination. A simple linear regression was run to assess the strength between \texttt{`Mean weekly stream hours’} (independent variable) and \texttt{`average viewers’} (dependent variable). Because of the non-linear relationship between the variables, the residuals were highly non-normal. A inverse (reciprocal) transformation of \texttt{`Average viewers’} was performed to correct for non-normality of the residuals. The transformation greatly improved the distribution of the residuals, making them nearly normal. The transformation could not correct for heteroskedasticity, but this is not an issue since inference is not being conducted.

\subsubsection{Model Specification}

The regression model is as follows:

\begin{equation}
\dfrac{1}{Average\ viewers_{i}} = \beta_{0} + \beta_{1} \ast Mean\ weekly\ stream\ hours_{i} 
\end{equation}
where \textit{i} is a Twitch account. \textit{`Average viewers'} is the average number of viewers that watched the respective Twitch account; and \textit{`Mean weekly stream hours'} is the number of hours that the respective Twitch account streamed over the year divided by 52.

\subsubsection{Results}\

%The regression results are reported below (Table \ref{table:reciprocal_regression_table}).

%%%%%%%%%%%%%%%%%%%%%%%%%%%%%
% Table: Regression table
%%%%%%%%%%%%%%%%%%%%%%%%%%%%%
\begin{table}[!ht]
  \centering % width=0.8\linewidth, height=0.375\textheight
  \includegraphics[scale=1]{../StatCrunch_Results/reciprocal/regression_table}
  \captionsetup{justification=raggedright, singlelinecheck=false, margin=2cm}
  \caption[Regression: Average Viewers Predicted by Stream Hours]{Simple linear regression model showing a moderate relationship between average viewers and mean weekly stream hours.}
  \label{table:reciprocal_regression_table}
\end{table}

R-squared is moderate, suggesting that the mean weekly stream hours can explain 56 percent of the variation in the average number of viewers. Because of the inverse transformation of the dependent variable, the signs of the intercept and mean weekly stream hours coefficient are reversed. This makes sense though, since an increase in the denominator of the equation (when the right hand side of the equation is back transform; Equation \ref{eq:back_transformed_reciprocal_eq}) will diminish the predicted number of average viewers.

%\begin{wraptable}{l}{0.5\textwidth} % Adjust the width as needed
%%    \centering
%    \includegraphics[width=\linewidth]{../StatCrunch_Results/reciprocal/regression_table}
%    \captionsetup{justification=centering, singlelinecheck=false, margin=0cm}
%    \caption[Regression: Average Viewers Predicted by Stream Hours]{Simple linear regression model showing a moderate relationship between average viewers and mean weekly stream hours.}
%    \label{table:reciprocal_regression_table}
%\end{wraptable}

Figure \ref{fig:stream_time_reciprocal_avg_viewers} shows the relationship between the mean weekly stream hours and the inverse of average viewers. The other scatter plots show the residuals. The histogram and Q-Q plots show that the residuals are now nearly normal, although excess kurtosis is still high, making the distribution of the residuals leptokurtic (skewness = -0.57036091; excess kurtosis = 6.0727871). Removing extreme dependent variable observations would help correct for this, but it is unlikely to be helpful since inference is not being conducted. Heteroskedasticity also was not corrected, but it also is not an issue as no hypothesis testing is being conducted, nor are confidence intervals being calculated. The model with coefficients is as follows:
\begin{equation}
 \dfrac{1}{Average\ viewers_{i}} = 4.04e^{-5} + 9.15e^{-6} \ast Mean\ weekly\ stream\ hours_{i} \label{eq:reciprocal_model_w_coef}
\end{equation}

\subsubsection{Scatter Plots}\

%%%%%%%%%%%%%%%%%%%%%%%%
% stream_time_reciprocal_avg_viewers
%%%%%%%%%%%%%%%%%%%%%%%%
\begin{figure}[H]
  \centering % width=\linewidth, height=\textheight
  \includegraphics[scale=0.5]{../StatCrunch_Results/reciprocal/stream_time_reciprocal_avg_viewers_plots}
  \captionsetup{justification=centering, singlelinecheck=false, margin=2cm}
  \caption[Average Viewers Predicted by Stream Hours]{Model plots. There residuals are symmetric, making the model a decent fit, nothwithstanding the non-constant variance.}
  \label{fig:stream_time_reciprocal_avg_viewers}
\end{figure}

Since one usually does not have a particular interest in knowing what the reciprocal of the average number of viewers is, \texttt{`Average viewers'} was reverse transformed and the transformed equation was plotted over the data on their original scale (Fig. \ref{fig:transformed_reciprocal_scatter}).


\begin{figure}[H]
  \centering % width=\linewidth, height=\textheight
  \includegraphics[width=0.8\linewidth]{../StatCrunch_Results/reciprocal/scatter_plot_reverse_transformed}
  \captionsetup{justification=centering, singlelinecheck=false, margin=2cm}
  \caption[Back Transformed Average Viewers Predicted by Stream Hours]{Back Transformed Average Viewers Predicted by Stream Hours. The fitted line is Equation \ref{eq:back_transformed_reciprocal_eq}}.
  \label{fig:transformed_reciprocal_scatter}
\end{figure}

\begin{equation}
Average\ viewers_{i} = \dfrac{1}{4.04e^{-5} + 9.15e^{-6} \ast Mean\ weekly\ stream\ hours_{i}} \label{eq:back_transformed_reciprocal_eq}
\end{equation}

\subsubsection{Discussion}\

Contrary to what one might think, the amount of time that a Twitch account streams has a negative assocaition with the average number of viewers. So if one wants to increase their viewership, streaming more does not seem to be a good strategy. A quick illustration is helpful. The median weekly stream hours for all accounts is 34.23 hours per week. Using Equation \ref{eq:back_transformed_reciprocal_eq}, we can estimate that the mean average number of viewers\footnote{The formulation appears awkward, but the variable is \texttt{`Average viewers'}, and the predicted value represents the mean \texttt{`Average viewers'} for that particular \texttt{x} value.} for accounts that stream the median number of hours per week is $\approx$ 2826. Accounts that stream ten hours less than median per week have a mean average viewership of 3813, which is 987 more viewers than accounts that stream the median number of hours per week (See Table \ref{tab:prediction_matrix} for predicted values).

\subsection{Stream Hours to Predict Followers}\

Do accounts that stream more have more followers? Intiutively, it would make sense. A simple linear regression was run to assess the relationship between streaming time, \texttt{`Mean weekly stream hours'}, and the number of \texttt{`Followers'} Twitch accounts have. The initial model revealed a highly non-linear relationship with highly non-normal residuals. A log transformation of \texttt{`Followers'} was conducted to correct for this and the models subsequently fit the data much better.

\subsubsection{Model Specification}\



\section{Multiple Linear Regression}

\subsection{Mature Accounts and Followers}\

How much does the difference in watch time explain the difference in the number of followers for accounts categorized as mature or not mature? Twitch accounts classified as \texttt{`Mature'} have fewer followers than those not classified as \texttt{`Mature'}. At the same time, mature accounts are watched fewer hours per week (Figure \ref{fig:box_plots_mature}).\footnote{Mature accounts also tend to have fewer peak and average viewers, and they also have not gained as many followers over the period. See Figure \ref{fig:box_plots_mature}.} While it is clear that mature accounts are watched less, it is interesting to ask how much that disparity in watch time can "explain"\footnote{This is observational data, so causal claims cannot be made here.} the disparity in the number of followers.

\subsubsection{Model Specificiation}\

Since \texttt{`Followers'} is highly skewed, the log of \texttt{`Followers'} was taken. A difference in means of the log followers was calculated. The difference in log means (-0.13351611) is exponentiated to get the ratio (0.875) between the two means (Equation \ref{eq:log_means_followers_mature}). Thus, mature Twitch accounts have 12.5 percent fewer followers than accounts that are not classified as mature.

\begin{equation}
\begin{aligned}
\mu_{1} & = \text{Mean of } \ln(\text{Followers}) \text{ where Mature} \\
\mu_{2} & = \text{Mean of } \ln(\text{Followers}) \text{ where not Mature} \\
\mu_{1} - \mu_{2} & = -0.13351611 \\
\emph{e}^{-0.13351611} & = 0.875
\end{aligned}
\label{eq:log_means_followers_mature}
\end{equation}

Next a multiple linear regression model was run to assess how much the differences between followers between mature and not mature accounts can be explained by differences in the number of watch hours. Log transformations on both the indepdendent (\texttt{`Mean weekly watch hours'}) and dependent variable (\texttt{`Followers'}) were conducted to correct for the non-linear nature of the relationship. The model is as follows:

\begin{equation}
ln(Followers_{i}) = \beta_{0} + \beta_{1} \ast Mature_{i} + \beta_{2} \ast ln(Mean\ weekly\ watch\ hours_{i})
\label{eq:multi_linear_model}
\end{equation}

where \emph{i} is each Twitch account in the dataset; \texttt{`Followers'} is the number of followers; \texttt{`Mature'} is a binary variable where 1 is when the account has been classified as mature, and 0 is when it has not; \texttt{`Mean weekly watch hours'} is the total watch time in minutes divided by 60 and divided by 52. Results are reported below.

\subsubsection{Results}\

\begin{table}[H]
  \centering % width=\linewidth, height=\textheight
  \includegraphics[scale = 0.7]{../StatCrunch_Results/followers_mature_watch_hrs/multi_regression_table}
  \captionsetup{justification=centering, singlelinecheck=false, margin=2cm}
  \caption[Multiple Linear Regression for Followers by Mature]{Followers regressed against Mature and Mean weekly watch hours}
  \label{tab:multi_regression_tab}
\end{table}

\subsubsection{Discussion}\
R-squared is rather weak (0.25), but the differences in watch time (mean weekly watch hours) does explain some of the differences observed between mature and non-mature accounts. The coefficient for \texttt{`Mature'} is -0.080855823. Exponentiating the coefficient ($\emph{e}^{-0.080855823}$) results in 0.9223. This is the ratio of followers for mature accounts to non-mature accounts when the natural log of mean weekly watch hours is held constant. In other words, mature accounts have 7.7 (1- 0.9223) percent fewer followers than non-mature accounts when controlling for differences in watch time.

\begin{figure}
    \centering
    \begin{subfigure}{0.45\textwidth}
        \centering
        \includegraphics[width=\textwidth]{../StatCrunch_Results/followers_mature_watch_hrs/residual_histogram}
        \caption{Histogram of residuals are nearly normal}
        \label{fig:plot1}
    \end{subfigure}
    \hfill
    \begin{subfigure}{0.45\textwidth}
        \centering
        \includegraphics[width=\textwidth]{../StatCrunch_Results/followers_mature_watch_hrs/residual_plot}
        \caption{No non-linear trends are detected in the data}
        \label{fig:plot2}
    \end{subfigure}
    \caption{Plots of residuals for the multiple linear regression model in Equation \ref{eq:multi_linear_model}}
    \label{fig:both_plots}
\end{figure}






%%%%%%%%%%%%%%%%%%%%%%%%%%
%%%%%%%%%%%%%%%%%%%%%%%%%%
%%%%%%%%    Appendix    %%%%%%%%%%%
%%%%%%%%%%%%%%%%%%%%%%%%%%
%%%%%%%%%%%%%%%%%%%%%%%%%%
\newpage

\section{Appendix}

\begin{figure}[H]
  \centering % width=\linewidth, height=\textheight
  \includegraphics[width=0.8\linewidth]{../StatCrunch_Results/mature/box_plots}
  \captionsetup{justification=centering, singlelinecheck=false, margin=2cm}
  \caption[Box Plots of Twitch Accounts by Mature Calssification]{Box Plots of Twitch Accounts by Mature Calssification}
  \label{fig:box_plots_mature}
\end{figure}



%%%%%%%%%%%%%%%%%%%
% Figure: Stream Hours Scatter Plot Matrix
%%%%%%%%%%%%%%%%%%%
\begin{figure}[H]
  \centering
  \includegraphics[width=0.795\linewidth]{../StatCrunch_Results/stream_scatter_plot_matrix.png}
  \captionsetup{justification=centering, singlelinecheck=false, margin=2cm}
  \caption[Stream Hours Scatter Plot Matrix]{Relationships between \texttt{`Mean weekly stream hours'} and other variables.}
  \label{fig:stream_scatter_matrix}
\end{figure}

%%%%%%%%%%%%%%%%%%%%%%%%%%%%%%%%%%%%%%%%%%%%
% Figure: Watch Hours Scatter Plot Matrix
%%%%%%%%%%%%%%%%%%%%%%%%%%%%%%%%%%%%%%%%%%%%
\begin{figure}[H]
  \centering
  \includegraphics[width=0.8\linewidth]{../StatCrunch_Results/watch_scatter_plot_matrix.png}
  \captionsetup{justification=centering, singlelinecheck=false, margin=2cm}
  \caption[Watch Hours Scatter Plot Matrix]{Relationships between \texttt{`Mean weekly watch hours'} and other variables.}
  \label{fig:watch_scatter_matrix}
\end{figure}

%%%%%%%%%%%%%%%%%%%%%%%%
% Figure: Followers Gained Scatter Plot Matrix
%%%%%%%%%%%%%%%%%%%%%%%%
\begin{figure}[H]
  \centering
  \includegraphics[width=0.8\linewidth]{../StatCrunch_Results/follow_gain_scatter_matrix}
  \captionsetup{justification=centering, singlelinecheck=false, margin=2cm}
  \caption[Followers Gained Scatter Plot Matrix]{Relationships between \texttt{`Followers gained'} and other variables.}
  \label{fig:follow_gain_scatter_matrix}
\end{figure}

%%%%%%%%%%%%%
% Figure: Histogram Matrix
%%%%%%%%%%%%%
\begin{figure}[H]
  \centering
  \includegraphics[width=0.8\linewidth]{../StatCrunch_Results/Histogram_Matrix.png}
  \captionsetup{justification=centering, singlelinecheck=false, margin=2cm}
  \caption[Histogram Matrix]{All distributions are heavily skewed and non-normal.}
  \label{fig:histogram_matrix}
\end{figure}


%%%%%%%%%%%%%
% Figure: Prediction Matrix
%%%%%%%%%%%%%
\begin{table}[H]
  \centering
  \includegraphics[scale=1]{../StatCrunch_Results/reciprocal/prediction_matrix}
  \captionsetup{justification=centering, singlelinecheck=false, margin=2cm}
  \caption[Prediction Matrix]{Predicted Values.}
  \label{tab:prediction_matrix}
\end{table}

% ceil(Rank(Mean weekly stream hours) / 90)

\end{document}